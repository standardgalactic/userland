\documentclass{standalone}
\usepackage{tikz}
\usetikzlibrary{shapes.geometric, arrows.meta, positioning}
\begin{document}
\begin{tikzpicture}
\end{center}

\section{Symmetric Monoidal Structure of Semantic Modules}
\label{sec:chapter8}

The monoidal structure enables parallel composition.

\subsection{Rationale}
Parallel composition enhances scalability, with $\otimes$ composing orthogonal flows.

\subsection{Anecdote}
In a data pipeline, $\otimes$ ensures orthogonality of preprocessing and inference modules.

\subsection{Foundations: Monoidal Categories}
Monoidal categories \cite{mac2013categories} have $\otimes$, $\mathbb{I}$, and isomorphisms satisfying coherence conditions.

\subsection{Monoidal Structure}
\[
M_1 \otimes M_2 = (F_1 \cup F_2, \Sigma_1 \times \Sigma_2, D_1 \sqcup D_2, \phi_1 \oplus \phi_2).
\]

\subsection{Proposition D.1: Associativity}
$(M_1 \otimes M_2) \otimes M_3 \cong M_1 \otimes (M_2 \otimes M_3)$.

**Proof**: Mac Lane’s coherence theorem ensures associativity \cite{lurie2009higher} (Appendix G).

**Natural Language Explanation**: The order of combining modules doesn’t affect the result, like mixing ingredients.

\begin{center}
\begin{tikzpicture}
  \node (M12) at (-2,0) {$(M_1 \otimes M_2) \otimes M_3$};
  \node (M23) at (2,0) {$M_1 \otimes (M_2 \otimes M_3)$};
  \draw[->] (M12) -- (M23) node[midway, above] {$\alpha$};
\end{tikzpicture}
\end{document}
