\documentclass{standalone}
\usepackage{tikz}
\usetikzlibrary{shapes.geometric, arrows.meta, positioning}
\begin{document}
\begin{tikzpicture}
\end{center}

\section{Stacks, Derived Categories, and Obstruction}
\label{sec:chapter5}

Stacks and derived categories handle complex merge obstructions.

\subsection{Rationale}
Sheaf gluing fails for higher-order conflicts. Stacks model these, ensuring coherence.

\subsection{Anecdote}
In federated AI, stacks resolve conflicting $\Phi$-fields across datasets.

\subsection{Foundations: Stacks and Derived Categories}
Stacks assign categories to $U$, with descent data. Derived categories $D(\mathcal{F})$ model obstructions via $\mathrm{Ext}^n(\mathbb{L}_M, \mathbb{T}_M)$ \cite{illusie1971complexe}.

\subsection{Obstruction Classes}
Obstructions are $\mathrm{Ext}^n(\mathbb{L}_M, \mathbb{T}_M)$, resolved by stacks aligning $\frac{\delta S}{\delta \Phi} = 0$.

\subsection{Historical Context}
Stacks and derived categories originated in algebraic geometry, with applications in type inference.

\subsection{Connections}
Chapter 4 provides sheaves, Chapter 2 informs obstructions, Chapter 6 uses $\mathrm{Ext}^n$.

\begin{center}
\begin{tikzpicture}
  \node (U) at (0,0) {$U$};
  \node (U1) at (-2,2) {$U_1$};
  \node (U2) at (2,2) {$U_2$};
  \node (U12) at (0,4) {$U_1 \cap U_2$};
  \node (S) at (0,6) {$\mathcal{S}(U)$};
  \draw[->] (U1) -- (S) node[midway, left] {$\mathcal{S}(U_1)$};
  \draw[->] (U2) -- (S) node[midway, right] {$\mathcal{S}(U_2)$};
  \draw[->] (U12) -- (U1) node[midway, left] {};
  \draw[->] (U12) -- (U2) node[midway, right] {};
\end{tikzpicture}
\end{document}
