\documentclass{standalone}
\usepackage{tikz}
\usetikzlibrary{shapes.geometric, arrows.meta, positioning}
\begin{document}
\begin{tikzpicture}
\end{center}

\section{Sheaf-Theoretic Modular Gluing}
\label{sec:chapter4}

Sheaf theory ensures local-to-global consistency in merges.

\subsection{Rationale}
GitHub’s merges ignore semantics, risking incoherence. Sheaves glue local modules into global structures, preserving RSVP dynamics.

\subsection{Anecdote}
In an AI project, developers fork a model. Sheaves ensure weight and architecture changes glue coherently.

\subsection{Foundations: Sheaf Theory}
Sheaf theory \cite{mac2013categories} models consistency. A sheaf $\mathcal{F}$ on $X$ satisfies:

\[
\mathcal{F}(U) \to \prod_i \mathcal{F}(U_i) \rightrightarrows \prod_{i,j} \mathcal{F}(U_i \cap U_j).
\]

Grothendieck topologies generalize this to sites.

\subsection{Theorem B.1: Semantic Coherence}
Let $\mathcal{F}$ assign $(\Phi, \vec{v}, S)$ to $U \subseteq X$. If fields agree on overlaps, there exists a unique global triple.

**Proof**: The equalizer condition ensures unique gluing via the limit $\mathcal{F}(X) \cong \varprojlim \mathcal{F}(U_i)$ \cite{mac2013categories} (Appendix G).

**Natural Language Explanation**: Sheaf gluing assembles a puzzle where pieces fit perfectly, forming a coherent picture.

\subsection{Module Gluing}
$\mathcal{F}$ assigns modules $M_i$, with gluing:

\[
M_i|_{U_i \cap U_j} = M_j|_{U_i \cap U_j} \implies \exists M \in \mathcal{F}(U_i \cup U_j).
\]

\subsection{Historical Context}
Sheaf theory originated with Leray, generalized by Grothendieck, with applications in distributed systems.

\subsection{Connections}
Chapter 3 provides objects, Chapter 2 informs gluing, Chapter 5 extends to stacks.

\begin{center}
\begin{tikzpicture}
  \node (U) at (0,0) {$U$};
  \node (U1) at (-2,2) {$U_1$};
  \node (U2) at (2,2) {$U_2$};
  \node (U12) at (0,4) {$U_1 \cap U_2$};
  \draw[->] (U) -- (U1) node[midway, left] {$\text{res}_1$};
  \draw[->] (U) -- (U2) node[midway, right] {$\text{res}_2$};
  \draw[->] (U1) -- (U12) node[midway, left] {$\text{res}_{12}$};
  \draw[->] (U2) -- (U12) node[midway, right] {$\text{res}_{21}$};
\end{tikzpicture}
\end{document}
