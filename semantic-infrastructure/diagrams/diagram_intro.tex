\documentclass{standalone}
\usepackage{tikz}
\usetikzlibrary{shapes.geometric, arrows.meta, positioning}
\begin{document}
\begin{tikzpicture}
\documentclass[12pt]{article}
\usepackage[utf8]{inputenc}
\usepackage{amsmath, amssymb, mathtools}
\usepackage{geometry}
\geometry{a4paper, margin=1in}
\usepackage{enumitem}
\usepackage{hyperref}
\usepackage{mathrsfs}
\usepackage{amsfonts}
\usepackage{bbm}
\usepackage{xcolor}
\usepackage{tikz}
\usetikzlibrary{shapes.geometric, arrows.meta, positioning}
\usepackage{tikz-cd}
\usepackage{listings}
\lstset{language=Haskell, basicstyle=\ttfamily\small, breaklines=true, frame=single}
\usepackage{lmodern}

\title{Semantic Infrastructure: Entropy-Respecting Computation in a Modular Universe}
\author{}
\date{August 2025}

\begin{document}

\maketitle

\begin{abstract}
This monograph establishes a rigorous framework for semantic modular computation, grounded in the Relativistic Scalar Vector Plenum (RSVP) theory, higher category theory, and sheaf-theoretic structures. Departing from syntactic version control systems like GitHub, we define a symmetric monoidal $\infty$-category of semantic modules, equipped with a homotopy-colimit-based merge operator that resolves divergences through higher coherence. Modules are entropy-respecting constructs, encoding functions, theories, and transformations as type-safe, sheaf-gluable, and obstruction-aware structures. A formal merge operator, derived from obstruction theory, cotangent complexes, and mapping stacks, enables multi-way semantic merges. The framework integrates RSVP field dynamics, treating code as flows within a semantic energy plenum. We propose Haskell implementations using dependent types, lens-based traversals, and type-indexed graphs, alongside blockchain-based identity tracking and Docker-integrated deployment. Formal proofs ensure well-posedness, coherence, and composability, with extensive diagrams visualizing categorical structures, field interactions, and topological tilings. This work provides a robust infrastructure for open, modular, intelligent computation where meaning composes, entropy flows, and semantic structure is executable.
\end{abstract}

\section{Introduction}
\label{sec:introduction}

Modern software development platforms, such as GitHub, are constrained by syntactic limitations that obstruct meaningful collaboration. Symbolic namespaces cause collisions, version control prioritizes textual diffs over conceptual coherence, merges resolve syntactic conflicts without semantic awareness, and forks fragment epistemic lineages. This monograph constructs a semantic, compositional, entropy-respecting framework, grounded in mathematical physics, higher category theory, and sheaf theory, to redefine computation as structured flows of meaning.

The Relativistic Scalar Vector Plenum (RSVP) theory models computation as dynamic interactions of scalar coherence fields $\Phi$, vector inference flows $\vec{v}$, and entropy fields $S$ over a spacetime manifold $M = \mathbb{R} \times \mathbb{R}^3$ with Minkowski metric $g_{\mu\nu} = \text{diag}(-1, 1, 1, 1)$. Semantic modules are localized condensates of meaning, integrated through thermodynamic, categorical, and topological consistency. The framework leverages higher category theory for modularity, sheaf theory for coherence, obstruction theory for mergeability, homotopy theory for higher merges, and type theory for implementation. This section outlines the monograph’s structure, with Chapters 1–14 developing the framework and Appendices A–G providing technical foundations, proofs, and diagrams.

\begin{center}
\begin{tikzpicture}
  \node[ellipse, draw] (RSVP) at (0,0) {RSVP Fields};
  \node[ellipse, draw] (Cat) at (4,0) {$\infty$-Category $\mathcal{C}$};
  \node[ellipse, draw] (Sheaf) at (2,2) {Sheaf $\mathcal{F}$};
  \node[ellipse, draw] (Merge) at (2,-2) {Merge $\mu$};
  \draw[->] (RSVP) -- (Sheaf) node[midway, left] {$\phi$};
  \draw[->] (Cat) -- (Sheaf) node[midway, right] {Objects};
  \draw[->] (Sheaf) -- (Merge) node[midway, left] {Gluing};
  \draw[->] (Merge) -- (Cat) node[midway, right] {Pushouts};
\end{tikzpicture}
\end{document}
