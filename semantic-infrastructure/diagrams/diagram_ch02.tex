\documentclass{standalone}
\usepackage{tikz}
\usetikzlibrary{shapes.geometric, arrows.meta, positioning}
\begin{document}
\begin{tikzpicture}
\end{center}

\section{RSVP Theory and Modular Fields}
\label{sec:chapter2}

RSVP models computation as dynamic entropy flows within a field-theoretic plenum, providing a foundation for semantic modules.

\subsection{Rationale}
RSVP redefines computation as a thermodynamic process, with code as semantic energy flows governed by $\Phi$, $\vec{v}$, and $S$, ensuring dynamic evolution and entropy minimization.

\subsection{Anecdote}
In a distributed AI system, modules for inference, training, and evaluation evolve independently. GitHub’s syntactic merges risk disruption, but RSVP aligns $\Phi$, $\vec{v}$, and $S$, ensuring coherence.

\subsection{Foundations: Field Theory and Stochastic PDEs}
Classical field theory (Faraday, Maxwell) models systems via fields over manifolds. Stochastic PDEs \cite{daprato2014stochastic} extend this to uncertain systems, formulated in Sobolev spaces $H^s(M)$:

\[
\|u\|_{H^s(M)}^2 = \int_M \sum_{|\alpha| \leq s} |\partial^\alpha u|^2 \, d^4x.
\]

RSVP fields evolve via:

\[
d\Phi_t = \left[ \nabla \cdot (D \nabla \Phi_t) - \vec{v}_t \cdot \nabla \Phi_t + \lambda S_t \right] dt + \sigma_\Phi dW_t,
\]

\[
d\vec{v}_t = \left[ -\nabla S_t + \gamma \Phi_t \vec{v}_t \right] dt + \sigma_v dW'_t,
\]

\[
dS_t = \left[ \delta \nabla \cdot \vec{v}_t - \eta S_t^2 \right] dt + \sigma_S dW''_t,
\]

over $M = \mathbb{R} \times \mathbb{R}^3$ with Minkowski metric.

\subsection{Theorem A.1: Well-Posedness of RSVP SPDEs}
Let $\Phi_t$, $\vec{v}_t$, $S_t$ evolve via the SPDEs with smooth initial conditions. The system admits a unique global strong solution in $L^2([0,T]; H^1(M))$, with conserved energy:

\[
E(t) = \int_M \left( \frac{1}{2} |\nabla \Phi_t|^2 + \frac{1}{2} |\vec{v}_t|^2 + \frac{1}{2} S_t^2 \right) d^4x.
\]

**Proof**: In $H = H^1(M) \times H^1(M)^3 \times H^1(M)$, the drift:

\[
F(\Phi, \vec{v}, S) = \begin{pmatrix}
\nabla \cdot (D \nabla \Phi) - \vec{v} \cdot \nabla \Phi + \lambda S \\
-\nabla S + \gamma \Phi \vec{v} \\
\delta \nabla \cdot \vec{v} - \eta S^2
\end{pmatrix},
\]

is Lipschitz. Noise terms are trace-class. A fixed-point argument ensures local existence, with global existence via a priori bounds. Itô’s formula shows $\mathbb{E}[dE(t)] = 0$ \cite{daprato2014stochastic} (Appendix G).

**Natural Language Explanation**: The proof ensures RSVP fields evolve smoothly, like a stable ecosystem, with energy balanced over time, enabling reliable computation.

\subsection{Module Definition}
A module $M = (F, \Sigma, D, \phi)$ is a sheaf section over $U \subseteq M$, with $\phi$ mapping to $(\Phi, \vec{v}, S)|_U$. Code induces:

\[
\Phi_f(x, t) = \Phi_1(x, t) + \int_0^t \vec{v}_f(\tau) \cdot \nabla \Phi_1(x, \tau) \, d\tau.
\]

\subsection{Historical Context}
RSVP builds on Maxwell’s field theory, Itô’s stochastic processes, and Hairer’s regularity structures \cite{hairer2014theory}.

\subsection{Connections}
Chapter 1 motivates RSVP, Chapter 3 constructs $\mathcal{C}$, Chapter 4 extends to sheaves. Appendix G provides the full proof.

\begin{center}
\begin{tikzpicture}
  \node[circle, draw] (Phi) at (0,0) {$\Phi_t$};
  \node[circle, draw] (v) at (2,0) {$\vec{v}_t$};
  \node[circle, draw] (S) at (1,-1.732) {$S_t$};
  \draw[->] (Phi) -- (v) node[midway, above] {$-\vec{v}_t \cdot \nabla \Phi_t$};
  \draw[->] (v) -- (S) node[midway, right] {$-\nabla S_t$};
  \draw[->] (S) -- (Phi) node[midway, left] {$\lambda S_t$};
  \draw[->] (S) -- (v) node[midway, below] {$\delta \nabla \cdot \vec{v}_t$};
  \draw[->] (Phi) -- (Phi) node[loop above] {$\nabla \cdot (D \nabla \Phi_t)$};
\end{tikzpicture}
\end{document}
