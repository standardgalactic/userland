\documentclass{standalone}
\usepackage{tikz}
\usetikzlibrary{shapes.geometric, arrows.meta, positioning}
\begin{document}
\begin{tikzpicture}
\end{center}

\section{Category-Theoretic Infrastructure}
\label{sec:chapter3}

Category theory provides a framework for semantic modularity, addressing GitHub’s limitations.

\subsection{Rationale}
GitHub obscures semantic relationships, necessitating a categorical framework where modules preserve intent.

\subsection{Anecdote}
In a bioinformatics collaboration, GitHub forces manual reconciliation. A categorical approach models modules in $\mathcal{C}$, preserving RSVP fields.

\subsection{Foundations: Higher Category Theory}
Category theory abstracts structures via objects and morphisms. Higher categories \cite{lurie2009higher} include higher morphisms, modeled via simplicial sets. Fibered categories support pullbacks, and functors preserve structure. Haskell’s type classes reflect categorical ideas.

\subsection{Module Category}
$\mathcal{C}$ is a symmetric monoidal $\infty$-category over $\mathcal{T}$, with objects $M = (F, \Sigma, D, \phi)$ and morphisms $f = (f_F, f_\Sigma, f_D, \Psi)$. The fibration $\pi : \mathcal{C} \to \mathcal{T}$ contextualizes modules.

\subsection{Historical Context}
Category theory’s applications \cite{lawvere2009conceptual} include denotational semantics. Fibered categories and $\infty$-categories provide precursors.

\subsection{Connections}
Chapter 1 motivates $\mathcal{C}$, Chapter 2 informs morphisms, Chapter 4 introduces sheaves.

\begin{center}
\begin{tikzpicture}
  \node (C) at (0,2) {$\mathcal{C}$};
  \node (T) at (0,0) {$\mathcal{T}$};
  \node (M1) at (-2,3) {$M_1$};
  \node (M2) at (2,3) {$M_2$};
  \node (T1) at (-2,1) {$T_1$};
  \node (T2) at (2,1) {$T_2$};
  \draw[->] (C) -- (T) node[midway, left] {$\pi$};
  \draw[->] (M1) -- (T1) node[midway, left] {$\pi$};
  \draw[->] (M2) -- (T2) node[midway, right] {$\pi$};
  \draw[->] (M1) -- (M2) node[midway, above] {$f$};
\end{tikzpicture}
\end{document}
