\documentclass{standalone}
\usepackage{tikz}
\usetikzlibrary{shapes.geometric, arrows.meta, positioning}
\begin{document}
\begin{tikzpicture}
\end{center}

\section{Multi-Way Merge via Homotopy Colimit}
\label{sec:chapter7}

Multi-way merges reconcile multiple forks.

\subsection{Rationale}
Pairwise merges risk incoherence. Homotopy colimits ensure higher coherence.

\subsection{Anecdote}
In an AI consortium, homotopy colimits align regional model forks.

\subsection{Foundations: Homotopy Theory}
A diagram $D : \mathcal{I} \to \mathcal{C}$ has homotopy colimit:

\[
\mathrm{hocolim}_\mathcal{I} D = \left| N_\bullet(\mathcal{I}) \otimes D \right|.
\]

\subsection{Theorem C.1 (Extended)}
If $\mathrm{Ext}^1 = 0$, $\mu(D) = \mathrm{hocolim}_\mathcal{I} D$ is unique.

**Proof**: The colimit is a derived pushout, with $\mathrm{Ext}^1 = 0$ ensuring existence \cite{lurie2009higher} (Appendix G).

\subsection{Implementation}
\begin{lstlisting}
data Diagram a = Diagram { nodes :: [Module a], edges :: [(Int, Int, Morphism a)] }
hocolim :: Diagram a -> Either String (Module a)
\end{lstlisting}

\subsection{Connections}
Chapter 6’s $\mu$ is generalized, Chapter 5 handles obstructions.

\begin{center}
\begin{tikzpicture}
  \node (M1) at (0,0) {$M_1$};
  \node (M2) at (2,0) {$M_2$};
  \node (M3) at (1,-1.732) {$M_3$};
  \node (M) at (1,2) {$M$};
  \draw[->] (M1) -- (M2) node[midway, above] {$f_{12}$};
  \draw[->] (M2) -- (M3) node[midway, right] {$f_{23}$};
  \draw[->] (M3) -- (M1) node[midway, left] {$f_{31}$};
  \draw[->, dashed] (M1) -- (M) node[midway, left] {$\iota_1$};
\end{tikzpicture}
\end{document}
