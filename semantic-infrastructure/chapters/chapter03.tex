\section{Sheaf-Theoretic Modular Gluing}
\label{sec:chapter4}

Sheaf theory ensures local-to-global consistency in merges.

\subsection{Rationale}
GitHub’s merges ignore semantics, risking incoherence. Sheaves glue local modules into global structures, preserving RSVP dynamics.

\subsection{Anecdote}
In an AI project, developers fork a model. Sheaves ensure weight and architecture changes glue coherently.

\subsection{Foundations: Sheaf Theory}
Sheaf theory \cite{mac2013categories} models consistency. A sheaf $\mathcal{F}$ on $X$ satisfies:

\[
\mathcal{F}(U) \to \prod_i \mathcal{F}(U_i) \rightrightarrows \prod_{i,j} \mathcal{F}(U_i \cap U_j).
\]

Grothendieck topologies generalize this to sites.

\subsection{Theorem B.1: Semantic Coherence}
Let $\mathcal{F}$ assign $(\Phi, \vec{v}, S)$ to $U \subseteq X$. If fields agree on overlaps, there exists a unique global triple.

**Proof**: The equalizer condition ensures unique gluing via the limit $\mathcal{F}(X) \cong \varprojlim \mathcal{F}(U_i)$ \cite{mac2013categories} (Appendix G).

**Natural Language Explanation**: Sheaf gluing assembles a puzzle where pieces fit perfectly, forming a coherent picture.

\subsection{Module Gluing}
$\mathcal{F}$ assigns modules $M_i$, with gluing:

\[
M_i|_{U_i \cap U_j} = M_j|_{U_i \cap U_j} \implies \exists M \in \mathcal{F}(U_i \cup U_j).
\]

\subsection{Historical Context}
Sheaf theory originated with Leray, generalized by Grothendieck, with applications in distributed systems.

\subsection{Connections}
Chapter 3 provides objects, Chapter 2 informs gluing, Chapter 5 extends to stacks.

\begin{center}

\subsection{Historical Context}
Obstruction theory extends cohomology, with applications in version control.

\subsection{Connections}
Chapter 5 handles obstructions, Chapter 4 provides context, Chapter 2 defines alignment.

\begin{center}
\end{lstlisting}

\subsection{Connections}
Chapter 6’s $\mu$ is generalized, Chapter 5 handles obstructions.

\begin{center}
\subsection{Rationale}
Tiling ensures a coherent semantic space, minimizing entropy.

\subsection{Anecdote}
In a knowledge graph, tiling ensures continuity of $\Phi$-fields.

\subsection{Foundations: Topological Dynamics}
An atlas $\{ U_i \}$ has transition maps $\Phi_i : U_i \to \mathcal{Y}$. Variational methods minimize:

\[
J(S) = \sum_{i,j} \|\nabla (S_i - S_j)\|^2.
\]

\subsection{Theorem E.1: Topological Tiling}
$X = \bigcup_i U_i$ admits a global $S : X \to \mathbb{R}$ minimizing $J(S)$.

**Proof**: Minimize $\mathcal{L}(S) = \sum_{i,j} \int_{U_i \cap U_j} |\nabla (S_i - S_j)|^2 \, d^4x$, yielding $\Delta S = 0$ \cite{evans2010partial} (Appendix G).

**Natural Language Explanation**: Tiling forms a mosaic with smooth transitions.

\begin{center}
\subsection{Rationale}
Embeddings map modules to $\mathbb{R}^n$, unlike GitHub’s search.

\subsection{Anecdote}
Embeddings reveal related drug discovery models.

\subsection{Foundations: Embedding Theory}
Embeddings use Gromov-Wasserstein distances:

\[
d_\Phi(M_1, M_2) = \|\Phi(M_1) - \Phi(M_2)\|_2.
\]

\subsection{Implementation}
$\Phi : \mathcal{M} \to \mathbb{R}^n$ embeds modules.

\begin{center}
\subsection{Thesis}
Code persists through composition, with $\Phi$, $\vec{v}$, $S$ as coherence, momentum, novelty.

\subsection{Connections}
Chapters 6–9 provide foundations.

